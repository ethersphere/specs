
Swarm involves a direct storage scheme of fixed size where chunks are stored on nodes with address corresponding to the chunk address.

The syncing protocols act in such a way that they reach those neighborhoods whenever a request is initiated.

Such a route is sure to exist as a result of the Kademlia topology of keep-alive connections between peers.

The process of relaying the request from the initiator to the storer is called forwarding and also the process of passing the chunk data along the same path is called backwarding.

Conversely - Backwarding and Forwarding are both notions defined on a keep alive network of peers as strategies of reaching certain addresses.

If we zoom into a particular node in the forwarding (or backwarding) path we see the following strategy:

- Receive a request
- Decide who to forward the request to (decision strategy)
- Have a way to match the the response to the original request.

The crucial step is the second one - strategy of choosing the peer to forward the request to and how they react to failure like stream closure or nodes dropping offline or closing the protocol connection and whether we proactively initiate several requests to peers.

The last step does not apply for the storer nodes, since they do not forward the request but they satisfy it.

The key elemement of these notions is that the decision about the next action is being done on the node level, which will select the next peers(s) and delegate them with handling the request.

The simplest representation of this would be a recursive algoritm that with every iteration gets closer to the target address and stops when it runs out of peers or successfully reaches the target node.


- We need a way to determine the "best" candidate peer to forward the request to, and if this option fails, continue with the "next-best" candidate until we exhaust available peers. The decision of picking the "best" peer is delegeate to an overlay driver that has the best knowledge of this peer's past history and performance and topology structure.
- We need a strategy of parallelisation of requests that we pass downstream, where appropriate. Parallel requests to different peers allow us increase the chances of successfully syncing the chunk but it comes with the cost of using our bandwidth allowance, so it's imperative to zoom in on an optimal balance between the two.
- We need a way to ensure that when we issue a syncing request we don’t end up in a situation when this request comes back around to us, wasting network resources.
- In the case when we are a "forwarder" node, we might consider a decision strategy on whether we want to cache the chunk in the event of a repeated request.
- To every forwarding/backwarding exchange we attach an incentivisation action that would take into account variables like the success of the action and the cost of performing the action.
- We need to design the optimal incentivisation scheme, to determine the optimal payment/settlement frequency and correctness of computation of the payment/charged amount. This also applies to both chunk storage scheme and the relayed request-response scheme.
- We need to have a sensible strategy when it comes to waiting for a peer to respond to our request; as a forwarder, we want to make our best effort to sync the chunk but without waiting for an excessive amount of time, which would lead to waste of resources.
- When receiving a response to an expired request - or we are unable to conclude if such a request has ever been issued - punishing measures should be imposed on the upstream peer.


Forwarding strategy refers to the process a node follows pertaining to relayed messages and responses in Forwarding Kademlia as relevant for chunk retrieval and upload.

Note that for retrieval as well as push-syncing, the protocol does not specify just allows forwarding of retrieve requests and chunk deliveries, respectively.
Whether a node forwards an incoming message and where it forwards them, should be guided by the incentives in order to attain stability.

Nodes  formulate and  advertise their prices for each PO bin separately. 
The optimal pricing strategy will reflect the PO bins in that the prices for closer bins will monotonically decrease. 

\subsubsection{Retries}
%disconnection

Both push-sync and retrieval messages are using backwarding (i.e., pass-back responses), and the compensation for forwarding only gets accounted for once the response is sent. This incentivises nodes to watch on peer connections where the downstream message was sent. In particular, if downstream peer disconnects  before the response in received, the forwarding should be repeated and tried on another peer.


%price-driven strategy

% acyclicity


 %\subsubsection{}

% multiple requests

\section{Routing with Kademlia}


Peer rating

When choosing a peer in relation to a given address - in addition to the distance between them - the Kademlia component will take into account two other factors:

- the historical performance of the given peer, both in terms of latencies and past occurences of protocol misalignments.
- the accounting aspect, peers with whom we have higher credit will be preferred.
- we should also prioritise those downstream peers that managed to produce responses in a previously computed amount of time (that would take into consideration the average time needed for a hop multiplied by the expected number of hops needed to reach a target neighborhood).

Kademila should be indexing peers by their proximity order and peers rating in order to prioritize peers based on their expected performance.

Decision strategy

An optimal decision strategy will take into account both proximity order and peer rating to select (out of all connected peers) the best one to pass down the request.

At the implementation level the Kademlia component will offer (in exchange for a given address) a stateful iterator that the client (protocol) will use to get the "next-best" peer.



\subsection{Kademlia}

Kademlia topology is a specific connectivity pattern used by all DISC protocols and its purpose is to route messages between nodes in a network using overlay addressing.

The message routing happens in such a fashion that with every network hop we will get closer to the target node, specifically at half of the distance covered by the previous hop.

Swarm uses the recursive/forwarding style of Kademlia.

This approach implies that every forwarding node - once it received a request - will keep an in-memory record that captures the request related information (requester, time of the request etc.) until the request is satisfied, rejected or times out.

Because a forwarder can not reliably tell how much time the downstream peer will need to satisfy the request - the choice of a resonable value for waiting period is a point of contention.

It is constrained by these factors:

- keeping the in-memory record for too long means that there's going to be a limit on how many concurrent requests a peer can keep "in-flight", because memory is limited.
- if the peer decides to time out prematurely (while downstream peers are still processing the request) then the effort of all the downstream peers will be wasted.
- we should distinguish between unsolicited chunks and chunks that we received from the downstream after we stopped waiting for the response (timed out). After a certain period of time all responses will be treated the same because of the need to free the allocated resources (the in-memory record).

Conversely the downstream should be informed when the upstream is no longer interested in the previously sent request, so it could free the used resources. This way the downstream won't return the chunk after the request has timed out, risking being punished.

\subsection{Incentivisation strategy}

An incentivisation strategy should be put in place in such way that it encourages honest collaboration between nodes.

This implies that a given peer will make the best effort to satisfy any request while not allowing any abuse and waste of its resources.

Having an accounting component that would keep track of the exchange activity between peers ensures that we do not allow excessive freeloading from the misbehaving peers.

Having a granular punishment strategy ensures that the peers who misbehave (perhaps due to network latencies) will not be sanctioned to the same extent as peers who engage in grave protocol breaches, but are given a chance to "clean up their act".
